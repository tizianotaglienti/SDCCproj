\documentclass[conference]{IEEEtran}
\IEEEoverridecommandlockouts
% The preceding line is only needed to identify funding in the first footnote. If that is unneeded, please comment it out.
\usepackage{cite}
\usepackage{amsmath,amssymb,amsfonts}
\usepackage{algorithmic}
\usepackage{graphicx}
\usepackage{textcomp}
\usepackage{xcolor}
\usepackage{hyperref}
\usepackage{svg}
\svgpath{{./figure/}}

\usepackage{listings}
\usepackage{color}

\definecolor{dkgreen}{rgb}{0,0.6,0}
\definecolor{gray}{rgb}{0.5,0.5,0.5}
\definecolor{mauve}{rgb}{0.58,0,0.82}

\lstset{frame=tb,
  language=Python,
  aboveskip=3mm,
  belowskip=3mm,
  showstringspaces=false,
  columns=flexible,
  basicstyle={\small\ttfamily},
  numbers=none,
  numberstyle=\tiny\color{gray},
  keywordstyle=\color{blue},
  commentstyle=\color{dkgreen},
  stringstyle=\color{mauve},
  breaklines=true,
  breakatwhitespace=true,
  tabsize=3
}

\def\BibTeX{{\rm B\kern-.05em{\sc i\kern-.025em b}\kern-.08em
    T\kern-.1667em\lower.7ex\hbox{E}\kern-.125emX}}
\usepackage[backend=biber]{biblatex}
\addbibresource{references.bib}


\begin{document}

\title{Sistemi Distribuiti e Cloud Computing
Progetto B3: Algoritmi di elezione distribuita}

\author{\IEEEauthorblockN{Tiziano Taglienti - 0304926}
\IEEEauthorblockA{\textit{Università degli studi di Roma "Tor vergata"}\\
tiziano.taglienti@alumni.uniroma2.eu}
}

\maketitle

\begin{abstract}
Gli algoritmi di elezione distribuita sono un'applicazione degli algoritmi di consenso distribuito, e hanno lo scopo di determinare un coordinatore in caso di crash del leader corrente.
Questi funzionano se metà dei nodi rimangono funzionanti.
In questo progetto vengono implementati l'algoritmo Bully e quello Ring-based di Chang e Roberts.
\end{abstract}

\begin{IEEEkeywords}
Ring-based, Bully, TCP, Docker, AWS, EC2, Python
\end{IEEEkeywords}


\section{Introduzione}\label{sec:intro}
Lo scopo del progetto è realizzare in \textbf{Python} un'applicazione distribuita che implementi gli algoritmi di elezione distribuita appena citati nell'abstract.
Si utilizzano dei container \textbf{Docker} per creare una rete decentralizzata di nodi e si esegue l'applicazione su un'istanza \textbf{EC2 con AWS}.

Nelle sezioni seguenti vengono descritti i servizi e le funzioni degli algoritmi, l'implementazione di questi ultimi e vengono discussi i test usati per valutare il funzionamento degli algoritmi.


\section{Servizi}\label{sec:services}


\subsection{Register}\label{register}

Il servizio \textbf{register} è necessario per la memorizzazione di tutti i processi che costituiscono la rete, associando un identificatore univoco a ognuno di essi.

Il server si comporta come una listening socket sulla porta TCP numero 1234 (vedi \textit{config.json}), in grado di accettare connessioni.
La socket rimane aperta per un tempo \textit{REG\_TIMEOUT}, al termine del quale viene inviata a tutti i processi una lista dei nodi nella rete, ordinati per identificatore crescente.
L'attribuzione dell'identificatore univoco ai nodi avviene dopo la fase di registrazione (inizialmente ogni nodo ha identificatore pari a \textit{DEFAULT\_ID}.
In seguito si stabilisce un coordinatore, cioè il processo che ha l'identificatore più grande. Per farlo, la variabile \textit{coordid} prende il valore dell'identificatore dell'ultimo nodo della lista e, con un messaggio di livello \textit{DEBUG}, questo valore viene stampato a schermo.

Durante la registrazione un nodo genera due socket: una viene usata per comunicare con il nodo register e l'altra serve per la ricezione di pacchetti da altri processi.


\subsection{Heartbeat}\label{heartbeat}

Il servizio di \textbf{heartbeat} ha come scopo principale quello di rilevare i crash e i fallimenti del coordinatore.

Finché c'è un'elezione in corso non si mandano messaggi di heartbeat; inoltre il coordinatore non invia messaggi di questo tipo.
I processi sfruttano questo servizio attraverso un thread che invia messaggi di heartbeat al coordinatore, attraverso una socket dedicata.
I messaggi di heartbeat vengono inviati periodicamente in base al valore di \textit{HB\_TIME} e, dopo aver inviato il messaggio, il thread aspetta per un tempo \textit{TOTAL\_DELAY}, dopo il quale si va in crash, per poi iniziare una nuova elezione.

Un coordinatore che non è fallito risponde ai messaggi di heartbeat con dei messaggi di acknowledgement.

\begin{lstlisting}
def heartbeat(self):
    while True:
        ...
        self.lock.acquire()

        if self.participant or (self.coordid in [self.id, constants.DEFAULT_ID]):
            self.lock.release()
            continue

        index = helpers.get_index(self.coordid, self.nodes)
        info = self.nodes[index]

        msg = helpers.message(self.id, Type['HEARTBEAT'].value, address[1], address[0])

        destination = (info["ip"], info["port"])

        try:
            hb_sock.connect(destination)
            hb_sock.send(msg)
            verbose.logging_tx(self.verb, self.logging, destination, (self.ip, self.port), self.id, eval(msg.decode('utf-8')))
            self.receive_ack(hb_sock, destination, constants.TOTAL_DELAY)
        # coord crash
        except ConnectionRefusedError:
            hb_sock.close()
            self.crash()
\end{lstlisting}

All'inizio di questo metodo è mostrato il modo in cui si definisce un lock per gestire le risorse condivise, dal momento che molti dati possono essere acceduti contemporaneamente da più thread.

\begin{figure}[htbp]
  \centering
  \includesvg[inkscapelatex=false, width = 245pt]{heartbeat.svg}
  \caption{Heartbeat service invoked by two nodes.}
\end{figure}


\section{Flag}\label{sec:flag}


\subsection{Verbose}\label{verbose}

Questa funzione consente di mostrare tutti i messaggi scambiati tra i processi, indicandone alcune informazioni, quali:
\begin{itemize}
	\item Timestamp del messaggio (in formato hh:mm:ss);
	\item Caratteristiche del nodo (indirizzo IP, numero di porta, identificatore);
	\item Mittente;
	\item Destinatario;
	\item Contenuto del messaggio.
\end{itemize}
Per offrire un'esecuzione \textbf{verbose} si specifica il flag \textit{-v} da linea di comando, e conseguentemente si inserisce un messaggio di livello \textit{DEBUG} sul logger.


\subsection{Delay}\label{delay}

Per provare il sistema in condizioni di maggiore stress, come richiesto nella specifica, è stato incluso un parametro \textit{delay} durante l'invio dei messaggi.
Il metodo corrispondente viene definito nel file \textit{helpers} e, in entrambi gli algoritmi, chiamato durante la fase di forwarding.
Il metodo consiste nella generazione di un tempo di attesa da parte del mittente per spedire il pacchetto.
Questa funzionalità è attivata automaticamente nei test.
Una possibile conseguenza è la scadenza del timeout del destinatario, che può dedurre un crash del processo da cui aspettava un pacchetto.


\section{Algoritmi}\label{sec:algo}


\subsection{Classi Algorithm e Type}\label{algorithm}

L'implementazione degli algoritmi si realizza attraverso una classe astratta \textit{Algorithm}, definita nell'omonimo file ed estesa dai due algoritmi di elezione distribuita.
Le classi che estendono \textit{Algorithm} devono fare override di tutti i metodi della classe di base, contrassegnati dal decoratore \textit{@abstractmethod}, quali:

\begin{lstlisting}
@abstractmethod
def start_election(self):
    pass

@abstractmethod
def answer(self):
    pass

@abstractmethod
def end(self, msg: dict):
    pass

@abstractmethod
def election(self, msg: dict):
    pass

@abstractmethod
def forwarding(self):
    pass
\end{lstlisting}

Nel file \textit{algorithm} compare la classe \textit{Type}, in cui sono indicati i sei tipi di messaggio scambiati tra i nodi:

\begin{lstlisting}
class Type(Enum):
    ELECTION = 0
    END = 1
    ANSWER = 2
    HEARTBEAT = 3
    REGISTER = 4
    ACK = 5
\end{lstlisting}

dove ANSWER è utilizzato solo nell'algoritmo \textbf{Bully}, e REGISTER solo durante la fase iniziale di registrazione.


\subsection{Inizializzazione}\label{init}

Entrambi gli algoritmi inizializzano un listening thread prima ancora di avviare l'elezione. Essendo un'operazione comune a entrambi, si trova nella classe Algorithm:

\begin{lstlisting}
def __init__(self, ...):
        ...
        
        thread = Thread(target = self.listening)
        thread.daemon = True
        thread.start()

        Algorithm.heartbeat(self)
\end{lstlisting}

Un processo è già eletto coordinatore alla fine della fase di register, quindi l'algoritmo inizia con i processi che inviano da subito dei messaggi di heartbeat al coordinatore, attendendo ACK.


\subsection{Chang-Roberts}\label{ring}

In questo caso, la topologia della rete è un anello orientato dove i messaggi sono inviati in senso orario.
Ogni processo conosce l'id degli altri e possiede un canale di comunicazione per il prossimo nodo dell'anello, quello con identificatore immediatamente maggiore del proprio.

L'algoritmo inizia con un coordinatore già stabilito e con gli altri nodi che gli mandano messaggi di heartbeat.
Quando il coordinatore va in crash inizia una nuova elezione, con l'invio di messaggi di tipo 0 (attraverso forwarding in senso orario) che contengono l'identificatore maggiore tra il proprio e quello indicato nel messaggio ricevuto. Quando questi due identificatori sono uguali, il ricevente si autoelegge coordinatore e invia un messaggio di tipo 1 per concludere l'elezione.

Per quanto riguarda il metodo \textit{forwarding}, innanzitutto si sceglie il processo successivo nell'anello accedendo al dizionario creato alla fine della registrazione, ordinato in base all'identificatore in maniera crescente.
Fatto questo, ci si connette a quel nodo e si invia il messaggio.

Quando ci si accorge di un fallimento del coordinatore (crash o scadenza di un timer associato all'heartbeat), si rimuove dalla lista dei nodi, così che gli altri non possano più interagire con lui.

\begin{figure}[htbp]
  \centering
  \includesvg[inkscapelatex=false, width = 245pt]{ring_1.svg}
  \caption{Election started by node with \textit{id=23} in Ring topology.}
\end{figure}


\subsection{Bully}\label{bully}

A differenza del caso precedente, ora si assume conoscenza e comunicazione completa tra i processi.
Si inizia sempre con un coordinatore e, quando questo crasha, il primo processo che si accorge del fallimento indice un'elezione.

L'elezione funziona in modo diverso: il nodo che l'ha ordinata manda un messaggio di tipo 0 ai soli processi con id maggiore del suo, aspettando risposte (messaggi ANSWER).
Se ne riceve almeno una, si disinteressa dell'elezione e l'algoritmo prosegue dai nodi che hanno risposto, altrimenti procede ad autoeleggersi coordinatore.
Nell'ultimo caso il processo invia a tutti i processi vivi dei pacchetti END.

In questo algoritmo, un timeout riguardante il coordinatore o un suo crash non porta alla rimozione del nodo dalla lista, poiché il fallimento non ha effetti sulla topologia della rete.

\begin{figure}[htbp]
  \centering
  \includesvg[inkscapelatex=false, width = 245pt]{bully_1.svg}
  \caption{Election started by node with \textit{id=23} using \textbf{Bully algorithm}.}
\end{figure}


\section{Test}\label{sec:tests}

Per provare il funzionamento degli algoritmi implementati sono stati eseguiti tre tipi di test:

\begin{enumerate}
    \item \textit{test\_any}: descrive il fallimento di un processo qualsiasi (escluso il coordinatore);
    \item \textit{test\_coord}: descrive il fallimento del coordinatore;
    \item \textit{test\_both}: descrive il fallimento di un processo qualsiasi e del coordinatore.
\end{enumerate}

In tutti i test, per interrompere un processo in ascolto su una specifica porta TCP si utilizza il metodo \textit{nodekill} della classe \textit{Utils}.
In questo metodo si sfrutta la libreria \textit{psutil} per filtrare i processi e ordinarli, per poi successivamente inviare un segnale di terminazione del nodo che ascolta su una porta di cui si specifica il numero. \footnote{Per fare ciò, bisogna avere i permessi di utente root, quindi è necessario eseguire cmd come amministratore.}

\begin{lstlisting}
def nodekill(self, port: int):
    for node in psutil.process_iter():
        for connections in node.connections(kind = 'inet'):
            if connections.laddr.port == port:
                try:
                    node.send_signal(signal.SIGTERM)
                except psutil.NoSuchProcess:
                    pass
                except psutil.AccessDenied:
                    pass
\end{lstlisting}

Per quanto riguarda l'esecuzione dei test, questa avviene in maniera interattiva: l'utente può decidere quale tipo di test eseguire, quale algoritmo utilizzare e il numero di processi che vengono creati (partendo da un minimo di quattro).


\section{Esecuzione}

L'applicazione può essere eseguita in due modi diversi:
\begin{enumerate}
    \item Esecuzione locale senza container \textbf{Docker};
    \item Esecuzione remota su un'istanza \textbf{EC2 con AWS} usando container \textbf{Docker}.
\end{enumerate}

Nel primo caso, basta l'esecuzione da linea di comando con gli appositi flag, sfruttando il file \textit{config.json} per gestire le impostazioni della rete.

Invece nel secondo caso, il programma viene sviluppato su un'istanza \textbf{EC2 con AWS}, dove ogni nodo viene eseguito su un container \textbf{Docker}.
Successivamente si sfrutta \textbf{Docker Compose}: tramite il file \textit{docker-compose.yml} si può utilizzare un singolo comando per creare e avviare tutti i servizi necessari, creando una rete privata all'interno della quale i container possono comunicare tra di loro.

\begin{figure}[htbp]\label{fig:arch}
  \centering
  \includesvg[inkscapelatex=false, width = 245pt]{arch.svg}
  \caption{Deployment using \textbf{AWS EC2} instance and \textbf{Docker} containers.}
\end{figure}


\section{Running Examples}

\begin{figure}[htb]\label{fig:register}
\includegraphics[width=\linewidth]{figure/register_terminal.png}
\caption{Register phase from three generic nodes and register node.}
\end{figure}

\begin{figure}[htb]\label{fig:verbose}
\includegraphics[width=\linewidth]{figure/verbose.png}
\caption{Example of the message showed by register node.}
\end{figure}

\begin{figure}[htb]\label{fig:tests}
\includegraphics[width=\linewidth]{figure/tests.png}
\caption{User interface to tests execution}
\end{figure}

\begin{figure}[htb]\label{fig:aws}
\includegraphics[width=\linewidth]{figure/aws_demo.png}
\caption{\textbf{Docker}, \textbf{Docker Compose} and application code info on an \textbf{AWS EC2} instance.}
\end{figure}
\printbibliography

% screen di: 
		-no % verbose durante l'esecuzione (inquadrare DEBUG)
		-% test quando eseguo (console principale) e cosa fanno i comandi
		-% test finishing
		-% fase di register (3 nodi che inviano richiesta al register che risponde (e basta, fino a qui))
		-% fase di heartbeat (messaggi 3 5 3 5 3)
		-% messaggio answer in bully (2)
		-no % fase di election

% per quanto riguarda aws (da mettere in "Esecuzione")
		% installo docker, docker compose, migro il codice e faccio vedere che ho installato
		% docker --version E docker-compose --version
		% screen quando scrivo LL per far vedere che c'è la cartella mia!!!!!

% :)



\end{document}